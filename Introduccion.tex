Least-squares support-vector machines (LS-SVM) for statistics and in statistical modeling,
are least-squares versions of support-vector machines (SVM),
which are a set of related supervised learning methods that analyze data and recognize patterns,
and which are used for classification and regression analysis.
In this version one finds the solution by solving a set of linear equations instead of a convex quadratic programming (QP) problem for classical SVMs.
Least-squares SVM classifiers were proposed by Johan Suykens and Joos Vandewalle.
[1] LS-SVMs are a class of kernel-based learning methods. 

En este proyecto se realizó el análisis de el método de Least-squares support-vector machines (LS-SVM) aplicado a problemas de control óptimo.
¿Pero que es este método?, ¿Cómo se aplica a problemas de control óptimo?, ¿Qué ventajas tiene respecto a otros métodos?.
El least squares support vector machine (LS-SVM) es una versión de las support vector machine (SVM) la cual es una técnica de aprendizaje supervisado que se utiliza para resolver problemas de clasificación y regresión.
Donde LS-SVM reformula el problema de optimización como una minimización de errores de mínimos cuadrados y utiliza ecuaciones lineales en lugar de las restricciones cuadráticas de las SVM tradicionales.
Además las restricciones en vez de ser desigualdades pasan a ser igualdades, mostrando así su caracter de regresión mas que de clasificación.

Así en este informe se presentará la formulación de LS-SVM y su aplicación a problemas de control óptimo, mostrando las ventajas y desventajas de este método respecto a otros métodos de control óptimo.
Más especificamente en tres problemas de control óptimo: manejo de la velocidad final de un carro cohete, estabilización a un punto de referencia en el sistema caótico del atractor de rossler y el pendulo invertido.
